\documentclass{article}
\usepackage{fontspec, fullpage}
\usepackage{polyglossia}
\usepackage{amsmath, amssymb, bbm, amsthm}
\setmainlanguage{hebrew}
\setmainfont{Times New Roman}
% \newfontfamily{\hebrewfont}{New Peninim MT}
\begin{document}
\title{תרגול 4 חשבון אינפיניטסימלי 1 שנת 2021/2}
\author{ישראל הבר}
\maketitle

\newtheorem{theorem}{משפט}
\newtheorem{exercise}{תרגיל}
\newtheorem{homeexercise}{תרגיל לבית}
\newtheorem{example}{דוגמה}
\theoremstyle{definition}
\newtheorem{definition}{הגדרה}
\newtheorem{notation}{סימון}
\newtheorem{claim}{טענה}
\newtheorem{comment}{\emph{הערה}}
\renewcommand\qedsymbol{$\blacksquare$}
\newcommand{\limtoinfty}{\underset{n\rightarrow\infty}{\lim}}
\newcommand{\goesto}{\underset{n\rightarrow\infty}{\longrightarrow}}
\newcommand{\goesfrom}{\underset{n\rightarrow\infty}{\longleftarrow}}

\begin{exercise}
הוכיחו כי לכל 
$a>0$
מתקיים כי 
\[\limtoinfty a^{\frac{1}{n}} = 1\]
\end{exercise}
\begin{proof}
נחלק למקרים - 
\begin{enumerate}
	\item $\mathbf{0<a<1}$ - \newline \newline
	נשים לב כי לכל 
$n\in\mathbb{N}$
מתקיים 
$0<a^{\frac{1}{n}}<1$
ולכן מהגדרת הגבול אנחנו צריכים להראות כי 
\[\forall\epsilon>0 \quad \exists N\in\mathbb{N}\quad \forall n\geq N: \quad 1 - a^{\frac{1}{n}}<\epsilon \]
עכשיו נבצע מניפולציות על האי-שיוויון -
\begin{align*}
1-a^{\frac{1}{n}}&<\epsilon \\
1-\epsilon&< a^{\frac{1}{n}} \\
\log_a(1-\epsilon)&>\frac{1}{n} \\
n&>\frac{1}{\log_a(1-\epsilon)} \\
\end{align*}
ולכן ניקח 
\[N=\max\left\{\left\lceil\frac{1}{\log_a(1-\epsilon)}\right\rceil, 1\right\} + 1\]
\item $\mathbf{a>1}$ - \newline \newline
	נשים לב כי לכל 
$n\in\mathbb{N}$
מתקיים 
$a^{\frac{1}{n}}>1$
ולכן מהגדרת הגבול אנחנו צריכים להראות כי - 
\[\forall\epsilon>0 \quad \exists N\in\mathbb{N}\quad \forall n\geq N: \quad a^{\frac{1}{n}}-1<\epsilon \]
עכשיו נבצע מניפולציות על האי-שיוויון -
\begin{align*}
a^{\frac{1}{n}} - 1&<\epsilon \\
a^{\frac{1}{n}}&< 1+\epsilon \\
\frac{1}{n}&<\log_a(1+\epsilon) \\
n&>\frac{1}{\log_a(1+\epsilon)} \\
\end{align*}
ולכן ניקח 
\[N=\max\left\{\left\lceil\frac{1}{\log_a(1+\epsilon)}\right\rceil, 1\right\} + 1\]
\end{enumerate}
\end{proof}

\begin{theorem}
נניח כי יש סדרות 
$a_n\goesto 0$
ו 
$b_n$
חסומה אז מתקיים כי 
$a_n\cdot b_n\goesto 0$
\end{theorem}
\begin{example}
ניקח 
\[a_n := \frac{1}{\ln (n+1)}, \quad b_n:=\sin(n) \]
מתקיימים בדיוק כל התנאים.
\end{example}
\begin{exercise}
הוכיחו כי אם 
$a_n$
סדרה מתכנסת והגבול הבא קיים 
\[\limtoinfty \frac{a_{n+1}}{a_n} = L\]
אזי מתקיים כי 
$|L|\leq 1$
\end{exercise}
\begin{proof}
נחלק שוב למקרים 
\begin{enumerate}
\item $\mathbf{a\neq 0}$
מתקיים כי 
\[|L| = \left|\limtoinfty \frac{a_{n+1}}{a_n}\right| = \left|\frac{\limtoinfty a_{n+1}}{\limtoinfty a_n}\right| = \left|\frac{a}{a}\right| = 1\leq 1\]
\item $\mathbf{a=0}$
מתקיים כי 
\[\forall\epsilon>0\quad \exists N\in\mathbb{N} \quad \forall n > N: \quad \left|\frac{a_{n+1}}{a_n} - L\right|>\epsilon\]
נניח בשלילה כי מתקיים 
$|L|>1$
מתקיים מאי שיוויון המשולש כי עבור כל אפסילון חיובי לבסוף מתקיים כי 
\[\left|\left|\frac{a_{n+1}}{a_n}\right|-|L|\right|<\epsilon\]
בפרט לבסןף מתקיים
\[\left|\frac{a_{n+1}}{a_n}\right|-|L|>-\epsilon\]
ולכן
\[\left|\frac{a_{n+1}}{a_n}\right|>|L|-\epsilon\]
ואם ניקח בפרט 
$\epsilon = |L|-1$
נקבל 
\[\left|\frac{a_{n+1}}{a_n}\right|>1\]
 מתקיים לבסוף כלומר סדרת הערכים המוחלטים עולה ממש, ובפרט סדרת הערכים המוחלטים לא מתכנסת ל0 בסתירה. 
\end{enumerate}
\end{proof}

\begin{exercise}
הוכיחו כי אם התנאים הבאים מתקיימים 
\[a_n\goesto a\neq0,\quad \frac{b_n}{a_n}\goesto 1\]
אזי מתקיים כי 
$b_n\goesto a$
\end{exercise}
\begin{proof}
\[\limtoinfty b_n = \limtoinfty a_n\cdot \frac{b_n}{a_n} = \limtoinfty a_n \cdot \limtoinfty \frac{b_n}{a_n} = 1\cdot a =a\]
\end{proof}

\begin{theorem}[משפט הסנדוויץ']
אם מתקיים כי 
\[a_n\goesto L,\quad b_n\goesto L\]
ולבסוף מתקיים 
\[a_n\leq c_n\leq b_n\]
אזי 
$c_n\goesto L$
\end{theorem}

\begin{example}
מצאו את הגבול הבא 
\[\sqrt[n]{2^n+3^n}\]
\end{example}
\begin{proof}
\[3\goesfrom 3 = \sqrt[n]{3^n}\leq \sqrt[n]{2^n+3^n}\leq \sqrt[n]{3^n+3^n} = \sqrt[n]{2\cdot 3^n} = \sqrt[n]{2}\cdot 3\goesto 3\]
ולכן ממשפט הסנדוויץ' הגבול הוא 3.
\end{proof}
\subsection*{התכנסות במובן הרחב}
\begin{definition}
נאמר כי סדרה מתכנסת במובן הרחב אם ורק אם מתקיים 
\[\forall M>0 \quad \exists N\in\mathbb{N}\quad \forall n>N: a_n>M\]
עם אותם סימונים של גבולות. ניתן גם לעשות את אותו דבר עם מינוס אינסוף.
\end{definition}
\begin{example}
הוכיחו לפי ההגדרה כי 
$2n^2\goesto \infty$
\end{example}

\begin{proof}
יהי 
$M>0$
לפי ההגדרה צריך להוכיח כי יש 
$N\in\mathbb{N}$
כך שלכל 
$n>N$
מתקיים 
$a_n>m$.
כלומר 
\begin{align*}
2n^2&>M  \left(\Longleftrightarrow\right)\\
n^2&>\frac{M}{2}  \left(\Longleftrightarrow\right)\\
n&> \sqrt{\frac{M}{2}} 
\end{align*}
ולכן ניקח 
\[N:=\left\lceil\sqrt{\frac{M}{2}} \right\rceil +1\]
\end{proof}

\begin{comment}
שימו לב כי אם 
$a_n$ 
סדרה מתכנסת לאינסוף ומתקיים כי 
$a_n\leq b_n$
אז גם 
$b_n$
מתכנסת לאינסוף.
\end{comment}
\begin{example}
מצאו 
\[\limtoinfty \sqrt[n]{n!}\]
\end{example}
\begin{proof}
נשים לב כי 
\[\left(\frac{n}{2}\right)^{n/2}\ \leq 1\cdot 2\cdot \dots\cdot \frac{n}{2}\cdot \dots\cdot n= n!\]
ולכן מתקיים
\[\infty\goesfrom \sqrt{\frac{n}{2}}= \sqrt[n/2]{\left(\frac{n}{2}\right)^{n/2}}\leq \sqrt[n]{n!}\]
ובפרט הסדרה מתכנסת לאינסוף.
\end{proof}
\begin{example}
מצאו את 
\[\limtoinfty \frac{n^n}{n!}\]
\end{example}
\begin{proof}
נשים לב כי
\[n!\leq \left(\frac{n}{2}\right)^{n/2}\cdot n ^{n/2}\]
ולכן 
\[\infty\goesfrom 2^{n/2} = \frac{n^n}{\frac{n^n}{2^{n/2}}} = \frac{n^n}{\left(\frac{n}{2}\right)^{n/2}\cdot n^{n/2}} \leq \frac{n^n}{n!} \]
ולכן הסדרה מתכנסת לאינסוף. 
\end{proof}

\begin{claim}
אם 
$a_n\goesto \infty$
ו 
$b_n$
חסומה מלרע אזי מתקיים 
\[a_n+b_n\goesto\infty\]
\end{claim}
\begin{proof}
ניקח את החסם מלרע 
$k\in\mathbb{R}$.
יהי 
$M>0$
עבור 
$M-k$
מתקיים כי יש 
$N\in\mathbb{N}$
 עבורו לכל
$n\geq N$
 מתקיים כי
$a_n\geq M-k$. ולכן יתקיים 
$a_n+b_n\geq M-k +b_n\geq M$
ולכן לפי הגדרת הגבול הסדרה מתכנסת לאינסוף.
\end{proof}

\begin{exercise}
הוכיחו כי
\[\frac{n^2}{n+1}\goesto \infty\]
נשים לב כי
\[\frac{n^2}{n+1}\geq \frac{n^2}{2n}\frac{n}{2}\goesto\infty\]
ולכן מסנדוויץ' הרחב נקבל כי הסדרה תתכנס לאינסוף.
\end{exercise}

נניח 
\[a_n,b_n\goesto \infty\]
מה הן האפשרויות עבור 
\[\limtoinfty \frac{a_n}{b_n}\]

\begin{enumerate}
\item $\mathbf{0}$ - $\frac{n}{n^2}$
\item $\mathbf{c}$
עבור 
$c>0$
למשל 
\[\frac{n}{2n} \goesto\frac{1}{2}\]
\item $\mathbf{\infty}$ - $\frac{n^2}{n}$
\item $\emptyset$ - 
למשל ניקח 
\[a_n:=\begin{cases}n, &\text{אי זוגי} \\ n^2, & \text{זוגי} \end{cases}, \quad b_n:=n^2\]
\end{enumerate}
\end{document}